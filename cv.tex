%% start of file `template.tex'.
%% Copyright 2006-2013 Xavier Danaux (xdanaux@gmail.com).
%
% This work may be distributed and/or modified under the
% conditions of the LaTeX Project Public License version 1.3c,
% available at http://www.latex-project.org/lppl/.


\documentclass[11pt,a4paper,sans]{moderncv}        % possible options include font size ('10pt', '11pt' and '12pt'), paper size ('a4paper', 'letterpaper', 'a5paper', 'legalpaper', 'executivepaper' and 'landscape') and font family ('sans' and 'roman')

% moderncv themes
\moderncvicons{awesome}


% casual (default),
% classic,
% oldstyle, and
% banking,
% and 7 color schemes:

% black,
% blue,
% green,
% grey,
% orange,
% purple, and
% red.
\moderncvstyle{banking}                             % style options are 'casual' (default), 'classic', 'oldstyle' and 'banking'
\moderncvcolor{red}                               % color options 'blue' (default), 'orange', 'green', 'red', 'purple', 'grey' and 'black'
%\renewcommand{\familydefault}{\sfdefault}         % to set the default font; use '\sfdefault' for the default sans serif font, '\rmdefault' for the default roman one, or any tex font name
%\nopagenumbers{}                                  % uncomment to suppress automatic page numbering for CVs longer than one page

% character encoding
\usepackage[utf8]{inputenc}                       % if you are not using xelatex ou lualatex, replace by the encoding you are using

% adjust the page margins
\usepackage[scale=0.75]{geometry}
%\setlength{\hintscolumnwidth}{3cm}                % if you want to change the width of the column with the dates
%\setlength{\makecvtitlenamewidth}{10cm}           % for the 'classic' style, if you want to force the width allocated to your name and avoid line breaks. be careful though, the length is normally calculated to avoid any overlap with your personal info; use this at your own typographical risks...

% personal data
\name{Viktor}{Holmberg}
% \title{Resumé title}                               % optional, remove / comment the line if not wanted
% \address{street and number}{postcode city}{country}% optional, remove / comment the line if not wanted; the "postcode city" and "country" arguments can be omitted or provided empty
\phone[mobile]{+4474~666~80~959}                   % optional, remove / comment the line if not wanted; the optional "type" of the phone can be "mobile" (default), "fixed" or "fax"
% \phone[fixed]{+2~(345)~678~901}
% \phone[fax]{+3~(456)~789~012}
\email{viktor.holmberg@gmail.com}                               % optional, remove / comment the line if not wanted
% \homepage{www.johndoe.com}                         % optional, remove / comment the line if not wanted
\social[linkedin][www.linkedin.com/pub/viktor-holmberg/49/790/4bb]{viktor holmberg}                     % optional, remove / comment the line if not wanted
% \social[twitter]{rutchkiwi}                             % optional, remove / comment the line if not wanted
\social[github]{rutchkiwi}                              % optional, remove / comment the line if not wanted
% \extrainfo{additional information}                 % optional, remove / comment the line if not wanted
% \photo[64pt][0.4pt]{picture}                       % optional, remove / comment the line if not wanted; '64pt' is the height the picture must be resized to, 0.4pt is the thickness of the frame around it (put it to 0pt for no frame) and 'picture' is the name of the picture file
% \quote{Some quote}                                 % optional, remove / comment the line if not wanted


%----------------------------------------------------------------------------------
%            content
%----------------------------------------------------------------------------------
\begin{document}
%-----       resume       ---------------------------------------------------------
\makecvtitle

\section{Profile}
Full stack developer with a backend focus. I like to get involved in all parts of development to understand the big picture. That includes everything from the database, to the frontend, to the people and processes outside the computer.

Coding-wise I value simplicity above all else, and when needed I’m not afraid to refactor my way there. 

I love to learn new tools and technologies, and believe in choosing the right tool for the job, keeping in mind that that right tool might not be the newest and shiniest thing.

My ideal work environment is open, collaborative and challenging.

\section{Education}
\cventry{5 yr}{Masters degree in Computer Science}{KTH Royal Institute of Technology}{Stockholm}{}{Masters track focus on machine learning and information retrieval}  % arguments 3 to 6 can be left empty
\cventry{6 mo}{Exchange studies}{University of Texas at Austin}{}{}{}

\section{Experience}
\cventry{8 mo}{Tech lead, energy finance team}{uSwitch}{London}{}{
uSwitchs commercial setup is complex, with each energy supplier having their own commercial model. As such, uSwitch is highly dependent on computer systems to estimate, invoice, and track revenue.
\begin{itemize}
    \item Ongoing work to ensure that we estimate revenue accurately, and get paid the right amount by our partners.
    \item Liaison between the accounting, invoicing, commercial and tech teams, making sure they all have access to the right information.
    \item Lead a rewrite of our system for attributing revenue. The old system had serious accuracy problems when working on incomplete data, which had made the users stop trusting it. 
    By using and showing exact error bounds, and doing comprehensive self-checking of the numbers, the new system was completely accurate. Using the new accurate data, we managed to identify several large payment problems, such as overpayments and underpayments from suppliers.
    We also made the new system stream based instead of batch based, which made the developer experience much more responsive.
    \item Lead the development of an automatic system for raising invoices. This system removed a large amount of manual labor for the team and ensured traceability of our invoicing process.
  \end{itemize}
}
% Uswitch is a price comparison website for utilites like gas and electricity.
% Uswitch's commerical setup is complex, with each energy supplier having their own commercial model. As such, uSwitch is highly dependant on computer systems to estimate, invoice, and track revenue.
% Liason between the accounting, invoicing, commercial and tech teams, making sure they all have access to the right information. 
% Ongoing work to ensure that we estimate revenue accuratly, and get paid the right amount by our partners.
% Lead a rewrite of our system for attributing revenue. The old system had serious accuracy problems when working on incomplete data, which had made its users lose trust in it. The old system "guessed" when some data was missing, which also made it extremely hard to debug problems, as it was always a bit off, even when it was working as expected. The fact that it was a one hour plus batch process did not help either.
% By using and showing exact error bounds, and doing comperhensive self-checking of the numbers, the new system was as good as completely accurate in it's calculations. Using the new accurate data, we managed to identify several large payment problems, such as overpayments and underpayments from suppliers.
% We also made the new system stream based instead of batch based, which made the developer experience much more responsive.
% Lead the development of an automatic system for raising invoices, removing a large amount of manual labour for the team and ensuring tracaility of our invoicing process.
\cventry{14 mo}{Back-end developer, energy back office team}{uSwitch}{London}{}{
\begin{itemize}
    \item Development on various backoffice systems with associated UI's.
    \item Ops work keeping microservices running smoothly with up to date tooling.
  \end{itemize}
}
\clearpage
\cventry{4 mo}{Full stack developer}{Depop}{London}{}{
\begin{itemize}
    \item Back-end development on highly concurrent payment system.
    \item Front-end development on backoffice tool for the support team.
  \end{itemize}
}
\cventry{18 mo}{Full stack developer}{Videoplaza}{Stockholm}{}{
\begin{itemize}
    \item Full-stack development on simulation-based ad inventory forecasting system;
    \begin{itemize}
      \item Enabled my team to do fully autonomous releases by separating our backend codebase into
  its own repository.
      \item Increased platform stability and team autonomy by separating the forecasting API into its
  own microservice.
      \end{itemize}
    \item Backend development on ad server:
      \begin{itemize}%
      \item Designed and implementing customer-friendly REST apis w. dynamic documentation using swagger.
      \end{itemize}
    \item Ops / CI work:
        \begin{itemize}
        \item Developed a solution for releasing the main legacy monolith without downtime.
        \item Moved forecasting system deployment from fabric scripts to ansible, enabling one-click, zero-downtime release for all forecasting microservices in both production and test environments.
        \item Reduced ad server build time from >1 hr to <20 mins in one days work.
        \end{itemize}
    \end{itemize}
    }

\cventry{6 mo}{Master thesis}{Findwise}{Stockholm}{}{Master thesis work developing an automated expert recommendation system, capable of inferring expertise of employees from data in a corporate internal network. 
\href{http://urn.kb.se/resolve?urn=urn:nbn:se:kth:diva-142450}{Read it \textbf{here}}
}

\cventry{1 yr (part time)}{Test automation developer}{Spotify}{Stockholm}{}{Development of graph-based test automation tools and automated tests for backend payment systems.
}


\section{Technologies}
\cvitem{Clojure} {Main language at uSwitch}
\cvitem{Java}{Main language for my work at Videoplaza.}
\cvitem{Python} {Main language for my work at Spotify}
\cvitem{Go} {Hobby projects}
\cvitem{Javascript}{Front-end development for Videoplaza and Depop. Includes tools and frameworks like angular, node, grunt, bower etc.}
\cvitem{Scala} {Main language at Depop, various small projects at Videoplaza. (Please don't make me work with Scala again!)}
\cvitem{Data technologies} {MySQL, PostgresSQL, RabbitMQ, Redis, Cassandra, Kafka.}
\cvitem{Operations/Infrastructure} {Drone CI, Jenkins, Docker, Terraform, Ansible, AWS (EC2, RDS, S3, ECS, cloudwatch, DynamoDB)}



\section{Fun facts}
\cvitem{Other job experience} {I've worked as a gardener, bus cleaner, security guard, and was in the swedish army for a year.}
\cvitem{Biggest fear} {Crane flies}
\cvitem{Favorite soda} {Fanta}
\clearpage
%-----       letter       ---------------------------------------------------------
% \recipient{Gocardless recruitment team}{}% \date{January 01, 1984}
% \opening{Hi!}
% \closing{}
% \makelettertitle

% I found out about you guys talking to Grey at Silicon milk roundabout. What got my attention straight away was that you guys are daringly using Elixir in production (favourite language at the moment). After working on a highly concurrent payment system in Scala I am eager to try something a bit more sane! :)

% I also gathered that you think code quality is important, which I like as someone who deeply values creating stuff that I can be proud of.

% It's probably worth mentioning that I don't have any ruby experience, so you'd have to be prepared that it'll be a few weeks before I'm fully up to speed. I love to learn new stuff and for me this is a great opportunity to learn how to code large systems in a dynamicly typed language.

% Looking forward to hearing from you!
% \makeletterclosing

% \clearpage
\end{document}
